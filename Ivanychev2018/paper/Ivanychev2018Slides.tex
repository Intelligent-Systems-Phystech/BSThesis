\documentclass{beamer}
    \usepackage[utf8]{inputenc}
    \usepackage[russian]{babel}
    \usepackage{amsmath,mathrsfs,mathtext}
    \usepackage{graphicx, epsfig}
    \usetheme{Warsaw}%{Singapore}%{Warsaw}%{Warsaw}%{Darmstadt}
    \usecolortheme{sidebartab}
    \usepackage{bbm}
    \DeclareMathOperator*{\argmin}{arg\,min}
    \definecolor{beamer@blendedblue}{RGB}{15,120,80}
    %----------------------------------------------------------------------------------------------------------
    \title[\hbox to 56mm{Построение суперпозиций\hfill\insertframenumber\,/\,\inserttotalframenumber}]
    {Построение суперпозиций \\ для прогнозирования временных рядов}
    \author[С.\,Д. Иванычев]{\large \\Сергей Дмитриевич Иванычев}
    \institute{\tiny
        Московский физико-технический институт\\
        Физтех-школа прикладной математики и информатики\\
        Факультет управления и прикладной математики\\
        Кафедра <<Интеллектуальные системы>>}

    \date{\footnotesize{Научный руководитель: д.ф.-м.н. В.В.~Стрижов}\\\vspace{\baselineskip}Выпускная квалификационная работа бакалавра\\\vspace{\baselineskip}Москва 2018}
    %----------------------------------------------------------------------------------------------------------
    \begin{document}

    %----------------------------------------------------------------------------------------------------------

    \begin{frame}
    %\thispagestyle{empty}
    \titlepage
    \end{frame}

    %-----------------------------------------------------------------------------------------------------

    \begin{frame}{Цель исследования}

    \vspace{3 mm}
    \begin{block}{\bf Задача}
    Выбор оптимальной прогностической системы моделей для краткосрочного прогнозирования объемов железнодорожных грузовых перевозок.\end{block}
    \begin{block}{\bf Цель}
    Построение прогностической модели, способной описывать временные ряды с асимметрично распределённым шумом.
    \end{block}
    \vspace{3 mm}
    \begin{block}{\bf Предложение}
    Использовать суперпозиции двух моделей.
    \end{block}

    \end{frame}

    % %----------------------------------------------------------------------------------------------------------

    % \begin{frame}{Литература}
    % \begin{itemize}
    % \item Журавлев Ю.~И., Рудаков К.~В., Корчагин А.~Д., Кузнецов М.~П., Мотренко А.~П., Стенина М.~М., Стрижов В.~В. Методы прогнозирования временных рядов на примере железнодорожных грузоперевозок // Вестник Российской академии наук. 2016.
    % \item А.П.~Мотренко, М.М.~Стенина, К.В.~Рудаков, М.П.~Кузнецов. Выбор оптимальной модели прогнозирования объёмов грузовых железнодорожных перевозок. 2015.
    % \item Н.Э.~Голяндина. Метод <<Гусеница>>-SSA: прогноз временных рядов, 2004.

    % \end{itemize}
    % \end{frame}

    % %----------------------------------------------------------------------------------------------------------

    % \begin{frame}{Постановка задачи}
    % \begin{block}{\bf Задача ретроспективного прогноза}
    % По известному отрезку ряда $\mathbf{x}=[x_1, ..., x_T]^\intercal$ построить прогноз $\hat{x}_{T+1}$ ряда в момент времени $T+1$.
    % \end{block}
    % \begin{block}{\bf Прогноз суперпозицией моделей}
    % $$\hat{x}_{T+1} = f\circ g(\mathbf{\hat{w}}, x_T, x_{T-1}, \ldots, x_1)$$
    % $$f, g\in \mathcal{F}$$
    % $\mathcal{F}$ -- семейство моделей-кандидатов, $\hat{w}\in\mathbb{R}^n$
    % \end{block}
    % \end{frame}

    % %----------------------------------------------------------------------------------------------------------

    % \begin{frame}{Критерий качества}
    % \begin{block}{\bf Предположения}
    % $$\varepsilon_{t} = x_t - \hat{x}_t,\quad\varepsilon_{t}\sim X, \quad \mathsf{E}(\varepsilon_t) \neq 0,\quad \mathsf{D}(\varepsilon_t) = \sigma^2$$
    % $X$ -- асимметричное распределение (пример: обратное распределение Гаусса).
    % \end{block}
    % \begin{block}{\bf Метрики качества прогноза}
    % $$\mathrm{MAPE} = \frac{1}{T}\sum_{t=1}^T \left|\frac{\varepsilon_t}{x_t}\right|, \quad \mathrm{MSE} = \frac{1}{T}\sum_{t=1}^T \varepsilon_t^2.$$
    % \end{block}
    % \begin{block}{\bf Оптимальная суперпозиция}
    % $$(f,g,\mathbf{\hat{w}})=\argmin_{f\in\mathcal{F},~g\in\mathcal{F},~\mathbf{\hat{w}}} \mathrm{MSE}(f\circ g(\mathbf{\hat{w}},x_T,x_{T-1},\ldots,x_1))$$
    % \end{block}
    % \end{frame}

    % %----------------------------------------------------------------------------------------------------------

    % \begin{frame}{Алгоритм блочного прогноза}

    % \begin{block}{\bf Алгоритм последовательного прогноза с накоплением}
    % $x_{T+1}$ считается по известному фрагменту ряда $x_1,\ldots,X_T$

    % $x_{T+k+1}$ считается по известному фрагменту ряда $x_1,\ldots,X_T$ и вычисленным значениям прогноза $x_{T+1},\ldots,x_{T+k}$

    % %$k$ называется \textit{запросом прогнозирования}.

    % \end{block}

    % \begin{block}{\bf Проблема}
    % Учёт качества прогнозирования на запросах, превышающих единицу, при выборе оптимальной суперпозиции.
    % \end{block}

    % \begin{block}{\bf Решение}
    % \begin{itemize}
    % \item Тестовая часть выборки разбивается на блоки длины $r$,
    % \item В пределах блока используется алгоритм прогнозирования с накоплением,
    % \item При переходе к следующему блоку значения предыдущего заменяются на истинные значения из тестовой выборки.
    % \end{itemize}
    % \end{block}

    % \end{frame}

    % %----------------------------------------------------------------------------------------------------------

    % \begin{frame}{Семейство моделей-кандидатов}
    % Состав семейства $\mathcal{F}$ моделей-кандидатов:
    % \begin{enumerate}
    % \item{Экспоненциальное сглаживание (параметр сглаживания $\alpha$),}
    % \item{Метод Кростена (параметр сглаживания $\alpha$),}
    % \item{Ядерное сглаживание (ядро, ширина окна),}
    % \item{Гусеница <<SSA>> (ширина окна, число спектральных компонент),}
    % \item{ARIMA (p,d,q),}
    % \item{Квантильная регрессия (функция штрафа),}
    % \item{LSTM (длина истории, эвристика).}
    % \end{enumerate}
    % В скобках указаны перебираемые по сетке структурные параметры.
    % \end{frame}

    % %----------------------------------------------------------------------------------------------------------

    % \begin{frame}{Зависимость качества от структурных параметров}
    % \makebox[\textwidth][c]{\includegraphics[width=\textwidth]{airline_10/hyperparameters}}
    % \end{frame}

    % %----------------------------------------------------------------------------------------------------------

    % \begin{frame}{Прогнозы основных моделей}
    % \makebox[\textwidth][c]{\includegraphics[width=\textwidth]{german_10/models_demo}}
    % \end{frame}

    % %----------------------------------------------------------------------------------------------------------

    % \begin{frame}{Матрица качества суперпозиций}
    % % FIXME: нормальная ошибка, а не log MSE
    % \makebox[\textwidth][c]{\includegraphics[width=\textwidth]{german_10/superposition}}
    % \end{frame}

    % %----------------------------------------------------------------------------------------------------------

    % \begin{frame}{Эмпирическая функция распределения ошибки}
    % \makebox[\textwidth][c]{\includegraphics[width=0.9\textwidth]{german_10/remainders}}
    % \end{frame}

    % %----------------------------------------------------------------------------------------------------------

    % \begin{frame}{Зависимость ошибки от горизонта прогнозирования}
    % % FIXME: линия для правила сломанной трости
    % \makebox[\textwidth][c]{\includegraphics[width=0.9\textwidth]{railroads_10/error_vs_horizon}}
    % \end{frame}


    % %----------------------------------------------------------------------------------------------------------

    % \begin{frame}{Зависимость ошибки от скоса распределения}
    % % FIXME: линия для правила сломанной трости
    % \makebox[\textwidth][c]{\includegraphics[width=0.9\textwidth]{error_vs_skew-eps-converted-to}}
    % \end{frame}

    % %----------------------------------------------------------------------------------------------------------

    % \begin{frame}{Результаты}
    % \begin{itemize}
    % \item Рассмотрены суперпозиции базовых алгоритмов (экспоненциальное сглаживание, метод Кростена, SSA, ARIMA, квантильная регрессия, LSTM).
    % \item Построены матрицы качества суперпозиций, распределение остатков модели, определён горизонт прогнозирования.
    % \item Показано, что использование суперпозиций может повышать качество прогноза.
    % \item Исследована зависимость ошибки от степени асимметричности распределения шумовых остатков: с ростом асимметричности суперпозиции достигают меньшей ошибки, чем базовые модели.
    % \end{itemize}
    % \end{frame}

    % %----------------------------------------------------------------------------------------------------------

    % \begin{frame}{Спасибо за внимание}
    % \end{frame}

    %----------------------------------------------------------------------------------------------------------

    \end{document}
